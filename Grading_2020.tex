% This is samplepaper.tex, a sample chapter demonstrating the
% LLNCS macro package for Springer Computer Science proceedings;
% Version 2.20 of 2017/10/04
%
\documentclass[runningheads]{llncs}
% A lot of package loading
\usepackage[pdftex]{graphicx}
\usepackage{geometry}
\usepackage[cmex10]{amsmath}
\usepackage{array, algpseudocode}
\usepackage{amsmath, amssymb, amsfonts, parskip, graphicx, verbatim}
\usepackage{url, hyperref}
\usepackage{bm, rotating, adjustbox, latexsym}
\usepackage{tabularx, booktabs}
\newcolumntype{Y}{>{\centering\arraybackslash}X}
\usepackage{float, setspace, mdframed}
\usepackage{color, contour, placeins, subfig, cite}
\usepackage[mathscr]{euscript}
\usepackage[osf]{mathpazo}
\usepackage{pgf, tikz, microtype, algorithm}
\usetikzlibrary{shapes,backgrounds,calc,arrows}
\usepackage{xcolor, colortbl, dsfont}


% If you use the hyperref package, please uncomment the following line
% to display URLs in blue roman font according to Springer's eBook style:
\renewcommand\UrlFont{\color{blue}\rmfamily}

\graphicspath{{figures/}}

\begin{document}
Submission of part 1 of the assignment is not graded. We are only going to give them \textit{constructive criticism} that they should take into consideration in their final submission.\\
For the format of the feedback, please use a standard pdf editor to add comments (e.g. acrobat reader). Don't use anything else but adding standard comments, highlighting and perhaps striking out text. Make sure that we can still edit these comments in acrobat reader!

\section{Overall}
Please pay attention to the following in the report:
\begin{itemize}
    \item Scientific writing style. Their report should be in the style of a scientific paper, so the language used should reflect this. Any time the language is too informal or unclear, add a note.
    \item Citations. Make sure they are using appropriate resources and citing them correctly. If there are no citations, make sure to mark this.
    \item Consistency of formulation. Keep an eye on the way certain variables are called, and make sure this remains consistent between different sections.
\end{itemize}

\section{Task-specific}


Please note the following in particular when grading these:
\begin{enumerate}
\item The pseudocode is the most important part of the assignments, this should match the original paper as precisely as possible
\item The pseudocode should follow the standard notation as specified in the template file. And other notation used should be explained in text and be used consistently.
\item The algorithm description should stand on its own, so someone who has not read the original paper should be able to get all of the working prinicples of the algorithm just from this description (combined with the pseudocode)
\item For the algorithm description, the animal metaphor can be used, but it should always be in quotations, e.g. 'bats', 'chickens',...
\item Any assumptions made should be clearly marked and explained. This also holds for places where the paper is vague and multiple interpretations are possible.
\item The more feedback you give, the better. Ideally, the feedback should make it so the students can easily correct their mistakes and implement a correct version of the algorithm.
\end{enumerate}

\end{document}